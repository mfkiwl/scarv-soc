
The formal verification flow is used to prove that critical design
properties hold true.
For the \SCARVSOC, this mostly takes the form of proving that bus
signalling protocols have been implemented correctly, and that
designer assumptions about things like one-hotness hold true.

Each module which uses the formal flow has the assertions included
inside the RTL source file.
The code implementing the assertions is always surrounded by a
pre-processor definition as shown in Figure \ref{fig:symbiyosys:module}.
By {\em ifdef}'ing the code, we can selectivley turn on individual
proofs within the entire SoC environment.
This means we can run proofs on the entire SoC design and still
keep the resource/compute requirements manageable.

\begin{figure}
\centering
\begin{lstlisting}[style=block, language=verilog]
module <module_name> (
// ... module ports ...
);
// ... module contents ...
`ifdef FORMAL_<module name>
// ... Formal assertions and assertions.
`endif
endmodule
\end{lstlisting}
\caption{Example verilog module showing how the design RTL is mixed
with preprocessor protected designer assertions.}
\label{fig:symbiyosys:module}
\end{figure}

Each individual proof job is described a {\tt .sby} file in
{\tt \$SOC\_HOME/verif/formal/}.
This file describes to the Symbiyosys tool which source files
should be included, and which compiler define to
use and activate the relevent proofs.

The following Make commands are used to run the formal flow.

\begin{itemize}

\item Clean up past proof results:

\begin{lstlisting}[style=block, language=verilog]
> make prove-clean
\end{lstlisting}

\item Run all available proofs:

\begin{lstlisting}[style=block, language=verilog]
> make prove-all
\end{lstlisting}

It is recommended to use Make's {\tt -j N } flag, where {\tt N} is the
number of jobs to run in parallel.
The following command will run as many jobs as your PC has CPU cores:

\begin{lstlisting}[style=block, language=verilog]
> make -j \$(nproc) prove-all
\end{lstlisting}

\item Run a specific proof:

\begin{lstlisting}[style=block, language=verilog]
> make prove-<job-name>
\end{lstlisting}

Where {\tt <job-name>} corresponds to one of the {\tt .sby} files in
{\tt \$SOC\_HOME/verif/formal/}, but with the extension removed..

\end{itemize}

The results for each proof job are placed in
{\tt \$SOC\_WORK/formal/<job name>}.


