
The following is a short guide / set of recommendations about building
software targeting the \SCARVSOC.

\begin{itemize}

\item You have $64$Kb of RAM to play with. Remember this.

\item While there is nothing stopping you from writing your own
    device drivers for peripherals on the SoC, it will be easier
    to link your program against the \SCARVSOC.
    Instructions for this can be found in \SECREF{sw:bsp}.

\item It will be simpler to run and debug your programs in a simulated
    environment rather than on actual hardware.
    Studying the selfchecking test and example program build and
    simulation flows will demonstrate how to do this.

\item The \SCARVCPU supports custom extensions for cryptography
    which cannot be accessed without a specialist toolchain.
    This can be obtained from
    \url{https://github.com/scarv/xcrypto}.

\item You will need to use a custom linker script so that your programs
    use the \SCARVSOC address map correctly.
    You can copy the example program linker script
    (\SOCDIR{src/examples/share/link.ld})
    for this.

\end{itemize}

