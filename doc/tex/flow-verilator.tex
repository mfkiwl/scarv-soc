
The Verilator flow is responsible for creating executable simulation
models of the \SCARVSOC inner subsystem.
It wraps the verilated model in a testbench, which emulates the
3rd party peripherals using C++ models.

\begin{itemize}
\item Makefiles and manifests are found in \SOCDIR{flow/verilator}.
\begin{itemize}
\item Looking inside the manifest files will show exactly which testbench
    and RTL files are included in the simulation models.
\end{itemize}
\item Testbench source code is found in \SOCDIR{verif/scarv-soc}.
\end{itemize}

Figure \ref{fig:flow:verilator:artifacts}
shows the main inputs and outputs of the Verilator flow in terms
of sources, compiler outputs and simulator outputs.

\begin{figure}
\centering
\includegraphics[width=0.8\textwidth]{verilator-srcs.png}
\caption{
    Flow diagram for building and running verilator based simulation
    models of the \SCARVSOC.
}
\label{fig:flow:verilator:artifacts}
\end{figure}


\subsubsection{A note on the Verilator makefile target}

All makefile targets which involve building the verilated model
are parameterised in terms of the RAM and ROM memory images which
are pre-loaded into the memory models, as well as the
output directory and executable.
This can be seen in the {\tt tgt\_verilator\_build} macro in
\SOCDIR{flow/verilator/Makefile.in}.

This is done because of tool quirks, and how verilator (dis-)allows
passing or overriding filepaths as Verilog parameters.
Each simulation target which uses a verilated model then creates
it's own target using the {\tt tgt\_verilator\_build} macro.
An example of this is visible in \SOCDIR{flow/selfcheck/Makefile.in}.

Ultimately, this detail is invisible when using the Make flow, as all
such dependencies are automatically resolved and built.
It is relevent however when adding your own build and simulation targets.

\subsubsection{Building the Verilated model}

\begin{itemize}
\item First, ensure that the getting started steps 
    (\SECREF{flow:getting-started}) have been followed.
    Your current working directory must be \SOCHOME.

\item Run the following make commands to build the generic verilated model.

\begin{lstlisting}[language=bash,style=block]
> make verilator-build
\end{lstlisting}

    This will build a generic simulation model, which expects as inputs
    two files: ``rom.hex" and ``ram.hex".
    These should be in the same working directory as the simulation
    executable when it is launched.
    The files {\em may} be empty, but they {\em must} exist.

    The generated model will be written too
    \SOCDIR[\SOCWORK]{verilator/verilated}.

\item The generic verilated model can be cleaned up using:
    
\begin{lstlisting}[language=bash,style=block]
> make verilator-clean
\end{lstlisting}

\end{itemize}

\subsubsection{Running the Verilated model}

\begin{itemize}

\item Given a built model at 
    \SOCDIR[\SOCWORK]{verilator/verilated},
    one can find the command line arguments the model takes by running:

\begin{lstlisting}[language=bash,style=block]
> ./work/verilator/verilated
./work/verilator/verilated [arguments]
	+q                            -
	+WAIT                         -
	+WAVES=<VCD dump file path>   -
	+TIMEOUT=<timeout after N>    -
	+PASS_ADDR=<hex addr>         -
	+FAIL_ADDR=<hex addr>         -
\end{lstlisting}

    where:
    \begin{itemize}
    \item {\tt +q} - Be quiet. Don't print so many messages.
    \item {\tt +WAIT} - Initialises the simulation and peripheral
        emulators, but waits for a keypress before running the simulation.
        This gives users time to attatch a terminal to the emulated UART
        port.
    \item {\tt +WAVES=<path>} - Where to dump a VCD waveform trace too.
        If this argument is omitted, then no waves are dumped.
        Care should be taken when using this argument over long periods,
        as VCD files grown in size very quickly.
        It is not recommended to use this argument when interracting
        with the program over UART.
    \item {\tt +TIMEOUT=<N>} - How many cycles to terminate the
        simulation after.
    \item {\tt +PASS\_ADDR=<addr>} - A hexadecimal address which, if
        an instruction is executed at, the simulation will stop and
        report a pass. Used by the selfchecking tests.
    \item {\tt +FAIL\_ADDR=<addr>} - A hexadecimal address which, if
        an instruction is executed at, the simulation will stop and
        report a failure. Used by the selfchecking tests.
    \end{itemize}

\item Running the following command will run the verilated model,
    and simulate execution of the first stage bootloader.

\begin{lstlisting}[language=bash,style=block]
> make verilator-run-waves
/home/ben/tools/verilator/bin/verilator --cc -CFLAGS "-O2" -O3 -CFLAGS -g -I/home/ben/scarv/repos/scarv-soc/extern/scarv-cpu/rtl/core --exe --trace --top-module scarv_soc   --Mdir /home/ben/scarv/repos/scarv-soc/work/verilator -o /home/ben/scarv/repos/scarv-soc/work/verilator/verilated -GBRAM_ROM_MEMH_FILE="\"rom.hex\"" -GBRAM_RAM_MEMH_FILE="\"ram.hex\"" -f /home/ben/scarv/repos/scarv-soc/flow/verilator/scarv-soc-rtl.manifest -f /home/ben/scarv/repos/scarv-soc/flow/verilator/scarv-soc-testbench.manifest
make -C /home/ben/scarv/repos/scarv-soc/work/verilator -f Vscarv_soc.mk
make[1]: Entering directory '/home/ben/scarv/repos/scarv-soc/work/verilator'
make[1]: Nothing to be done for 'default'.
make[1]: Leaving directory '/home/ben/scarv/repos/scarv-soc/work/verilator'
cp /home/ben/scarv/repos/scarv-soc/work/fsbl/fsbl.hex /home/ben/scarv/repos/scarv-soc/work/verilator/rom.hex
cp /home/ben/scarv/repos/scarv-soc/work/fsbl/fsbl.hex /home/ben/scarv/repos/scarv-soc/work/verilator/ram.hex
cd /home/ben/scarv/repos/scarv-soc/work/verilator/ && /home/ben/scarv/repos/scarv-soc/work/verilator/verilated  \
        +WAVES=/home/ben/scarv/repos/scarv-soc/work/verilator/waves.vcd \
        +TIMEOUT=10000
> /home/ben/scarv/repos/scarv-soc/work/verilator/verilated +WAVES=/home/ben/scarv/repos/scarv-soc/work/verilator/waves.vcd +TIMEOUT=10000 
>> Dumping waves to: /home/ben/scarv/repos/scarv-soc/work/verilator/waves.vcd
>> Timeout after 10000 cycles.
>> Open Slave UART at: /dev/pts/5
__FILE__:__LINE__ - UART Register not implemented.
>> Finished after 10000 simulated clock cycles
>> TIMEOUT
\end{lstlisting}

    A VCD wave dump will be written to \SOCDIR[\SOCWORK]{verilator/waves.vcd}.

\end{itemize}

\subsubsection{Using the emulated UART}
\label{sec:flow:verilator:uart}

The verilator testbench contains an emulated UART peripheral, which
models the behaviour of the Xilinx UARTLite IP in the SoC outer layer.
The model creates a unix ``psuedo-terminal", which we can connect
a terminal application like PuTTY or miniterm too, and then interract with
as if talking to the real implemented device.
For this example, we will show how to ``boot" the SoC using the first stage
boot loader, upload a program too it over the uart and see that program
executing.

\begin{itemize}

\item First, ensure the getting started steps in
    \SECREF{flow:getting-started}
    have been followed.

\item Build the verilated model, and move to the directory containing
    the executable.

\begin{lstlisting}[language=bash,style=block]
> # Terminal 1
> make verilator-build
> cd \$SOC_WORK/verilator
\end{lstlisting}

\item Run the following command to start the model, but pause it just before
    the simulation starts running.

\begin{lstlisting}[language=bash,style=block]
> # Terminal 1
> ./verilated +WAIT +TIMEOUT=100000000
>> Timeout after 100000 cycles.
>> Open Slave UART at: /dev/pts/5
>> Press return to begin simulation:
\end{lstlisting}

\item Open your favourite serial console app, and connect it to
    the port indicated on the ``{\tt Open Slave UART at:}" line.
    In this case, we can open a {\em new} terminal, and run miniterm:

\begin{lstlisting}[language=bash,style=block]
> # Terminal 2
> miniterm.py /dev/pts/5
\end{lstlisting}

\item Switch back to the first terminal session, and press return.
    The serial application should print the following message,
    inidicating that the bootloader has completed and is waiting to
    download a program to execute.

\begin{lstlisting}[language=bash,style=block]
> # Terminal 2
--- Miniterm on /dev/pts/5  9600,8,N,1 ---
--- Quit: Ctrl+] | Menu: Ctrl+T | Help: Ctrl+T followed by Ctrl+H ---
scarv-soc fsbl
\end{lstlisting}

    Eventually, the simulation will timeout, closing the serial port
    and likely angering your serial application.

\end{itemize}

\noindent
Now we know the bootloader works, lets download a proper program
and interract with the CPU.

\begin{itemize}

\item
    First, build the ``UART" example program:

\begin{lstlisting}[language=bash,style=block]
> # Terminal 1
> cd \$SOC_HOME
> make example-uart
\end{lstlisting}

    This will place the build software in
    \SOCDIR[\SOCWORK]{examples/uart}.
    Many files in there are not needed for this example, since we're doing
    the hard way what the make flow can wrap up for us.
    For this example, we just need the
    \SOCDIR[\SOCWORK]{examples/uart/uart.bin}
    file.

\item Next, start the simulation model as before:

\begin{lstlisting}[language=bash,style=block]
> # Terminal 1
> make verilator-build
> cd \$SOC_WORK/verilator
> ./verilated +WAIT +TIMEOUT=100000000
>> Timeout after 100000 cycles.
>> Open Slave UART at: /dev/pts/5
>> Press return to begin simulation:
\end{lstlisting}

\item In a {\em new} terminal, run the following command from the
    \SOCHOME directory. Note that the {\tt /dev/pts/5} part will
    likely be different for each machine.

\begin{lstlisting}[language=bash,style=block]
> # Terminal 2
> ./bin/upload-program.py /dev/pts/5 \$SOC_WORK/examples/uart/uart.bin 0x20000000 
Waiting for FSBL, please reset the target...
Programming /home/ben/scarv/repos/scarv-soc/work/examples/uart/uart.bin, size=300, start=0x20000000
Programming: Done
\end{lstlisting}

    This program tells the bootloader to place the {\tt uart.bin}
    file at memory address {0x20000000} and then jump to that
    address to run the program.

    Initially, the command will hang after printing ``Waiting for FSBL...".
    At this point, you should press return in the terminal which is running
    the model.

\item We can now re-connect to the serial port and start interracting
    with the program.

\begin{lstlisting}[language=bash,style=block]
> # Terminal 2
> miniterm.py /dev/pts/5
--- Miniterm on /dev/pts/5  9600,8,N,1 ---
--- Quit: Ctrl+] | Menu: Ctrl+T | Help: Ctrl+T followed by Ctrl+H ---
UART Echo Example Program
Hello world all the
kings horses and all the kings men.
Recieved Finish Byte. Stop.
\end{lstlisting}
    
    The UART is just a simple echo program.
    Anything you type is just echoed back by the CPU.
    The UART program source code is found in \SOCDIR{src/examples/uart}.

\item

    Sending a ``{\tt !}" character will tell the CPU to terminate
    the program.

\end{itemize}

\noindent
Because this is so long winded, it's been wrapped up in a make command.

\begin{itemize}
\item With the simulator model waiting for a keypress, run:

\begin{lstlisting}[language=bash,style=block]
> # Terminal 2
> make xilinx-interract PORT=<port name>
\end{lstlisting}
    
    Once you see the ``Waiting for FSBL..." message, press return
    on the simulator terminal, and the flow will automatically
    upload the example UART program and dump you into a serial
    terminal.

\item The source for this flow is found in
    \SOCDIR{flow/xilinx/Makefile.in}.

\item Exactly the same flow can be used to upload a program to an
    FPGA via the UART port.
    The only thing that changes is the {\tt /dev/*} path passed in.

\end{itemize}

