
The \SCARVSOC implements the following functionality:

\begin{itemize}

\item  The SoC will be implementable primarily on an FPGA.

\item The SoC will be implementable on an ASIC with minimal changes.

\item The number of FPGA specific blocks within the design will be minimised.
    Initially, some vendor specific IP blocks may be used for bring-up, but
    these should be replaced with free/open-source components eventually.

\item The SoC will use the SCARV CPU
    \footnote{\url{https://github.com/scarv/scarv-cpu}}
    core.

\item The CPU core will be augmented with the XCrypto
    \footnote{\url{https://github.com/scarv/xcrypto}}
    instruction set extensions.

\item The CPU core will be capable of running the Zephyr
    \footnote{\url{https://zephyrproject.org}}
    RTOS.

\item The SoC will have access to 64Kb of on-chip SRAM.

\item The SoC will support several common peripherals:

\begin{itemize}

\item UART Serial Interface.

\item 16 GPIO Pins.

\item One external interrupt pin.
\item I2C Serial Interface.
\item SPI flash controller.
\item Standalone SPI peripheral controller.

\item The SoC will optionally support a network interface (Ethernet / WiFi)

\end{itemize}

\item This will be based on existing open source or vendor specific designs.

\item The SoC must fit into a Artix-7 Xilinx FPGA, running at 100MHz.


\end{itemize}

