\documentclass{scarv-report}

\title{SCARV-SoC\\Technical Report and User Guide}
\date{Version $0.0.1$ (\today)}
\author{Ben Marshall}
\affil{
Department of Computer Science, University of Bristol,\\
Merchant Venturers Building, Woodland Road,\\
Bristol, BS8 1UB, United Kingdom.\\
\url{{ben.marshall}@bristol.ac.uk}
}

\begin{document}

% =============================================================================

\MKPROLOGUE

% =============================================================================

\section{Introduction}
\label{sec:intro}

\import{./tex/}{intro.tex}

\section{Design Requirements}
\label{sec:intro}

\import{./tex/}{requirements.tex}

\section{Project Organisation}
\label{sec:intro}

\import{./tex/}{organisation.tex}

\section{SoC Design}

\subsection{SCARV-CPU}
\import{./tex/}{design-scarv-cpu.tex}

\subsection{Local Interconnect}
\import{./tex/}{design-local-ic.tex}

\subsection{Local Memories}
\import{./tex/}{design-local-mem.tex}

\subsection{AXI Bus Bridge}
\import{./tex/}{design-axi-bridge.tex}

\section{Hardware Development Flows}

\subsection{Make Flow Overview}
\import{./tex/}{flow-make.tex}

\subsection{Verilator Simulation}
\import{./tex/}{flow-verilator.tex}

\subsection{Symbiyosys Formal Verification Flow}
\import{./tex/}{flow-symbiyosys.tex}

\subsection{Yosys Synthesis}
\import{./tex/}{flow-yosys-synth.tex}

\subsection{Xilinx Vivado Project}
\import{./tex/}{flow-xilinx-vivado.tex}

\subsection{Bootloader Usage}
\import{./tex/}{flow-bootloader.tex}

\section{Software Development Guide}

\subsection{First Stage Boot Loader}
\import{./tex/}{sw-fsbl.tex}

\subsection{Board Support Package}
\import{./tex/}{sw-bsp.tex}

\subsection{Example Programs}
\import{./tex/}{sw-examples.tex}

\subsection{Developing Programs}
\import{./tex/}{sw-development.tex}

% =============================================================================

\MKEPILOGUE

% =============================================================================

\end{document}

