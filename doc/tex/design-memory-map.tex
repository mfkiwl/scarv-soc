
The SoC memory map is split into two levels.
The first level is managed by the the local memory interconnect
(\SECREF{design:block:local-ic}),
and is detailed in table \ref{tab:memory-map:1}.

Requests which map to an inner layer peripheral are handled completely
by the local memory interconnect.
Requests which map to the AXI Bus Bridge memory space are forwarded to
the outer layer AXI interconnect and the on to the relevent 3rd party
peripheral.
The current Xilinx specific AXI interconnect memory map is shown in
table \ref{tab:memory-map:2}.

Requests which map to an empty part of the first level address space
are handled by the local memory interconnect.
It is possible for requests map to a valid address in the first
level, but be invalid in the second level.
In this case, the error response is handled by the
outer layer memory interconnect.

\begin{table}[H]
\centering
\begin{tabular}{lll}
Address      & Range        & Description   \\ \hline
{\tt 0x4000\_0000} & {\tt 0x4FFF\_FFFF} & AXI Bus Bridge                    \\
{\tt 0x2000\_0000} & {\tt 0x2000\_FFFF} & 64K RAM                           \\
{\tt 0x1000\_0000} & {\tt 0x1000\_3FFF} & First Stage Boot Loader 1K ROM    \\
{\tt 0x0000\_0000} & {\tt 0x0000\_0FFF} & CPU Internal memory mapped devices.
\end{tabular}
\caption{
    First stage \SCARVSOC memory map, managed by the local memory
    interconnect.
}
\label{tab:memory-map:1}
\end{table}

\begin{table}[H]
\centering
\begin{tabular}{lll}
Address      & Range        & Description  \\ \hline
{\tt 0x4000\_2000} & {\tt 0x4000\_20FF} & AXI mapped GPIO peripheral.  \\
{\tt 0x4000\_1000} & {\tt 0x4000\_10FF} & AXI mapped UART peripheral.  \\
\end{tabular}
\caption{
    Second stage \SCARVSOC memory map, managed by the outer layer
    AXI interconnect.
}
\label{tab:memory-map:2}
\end{table}

Note that the size of the RAM may be configured by twiddling the
{\tt BRAM\_RAM\_SIZE} top level parameter of the {\tt scarv\_soc}
module.
This will auto-magically change the address mappings and signal widths.

