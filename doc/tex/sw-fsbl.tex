
The First Stage Bootloader (FSBL) is used to download programs
into the local RAM.
It executes exclusively from the local ROM.
Source code for the FSBL is kept in
\SOCDIR{src/fsbl}.
Operation of the FSBL is as follows.

\begin{enumerate}

\item When the SoC is first reset, the initial CPU Program Counter value
    points to the base address of the ROM.
    The ROM contains the FSBL, which is the first thing executed.

\item Initially, the FSBL zeros all general purpose registers and
    disabled interrupts globally.
    This is done in the {\tt boot.S} file.

\item It then initialises the stack pointer to the the top of the RAM,
    and jumps from the {\tt boot.S} file  to the {\tt fsbl\_main} function
    in {\tt fsbl.c}.

\item The C code prints a welcome message to indicate the bootloader is
    active: \\ ``{\tt scarv-soc fsbl\textbackslash n}".

\item It then waits for a $4$-byte, little-endian sequence from the
    UART, indicating the size of the program to be downloaded in bytes.

\item Next, it waits for another $4$-byte little endian sequence indicating
    {\em where} in the memory space to dump the downloaded program.

\item It then waits until the requisite number of bytes are downloaded
    and stored at the correct memory address.

\item Once the entire program is downloaded, it jumps to the start address
    of the program. I.e. it's start location in memory.

\item The new program {\em should not} depend on the stack pointer or
    to of the RAM memory to be zero'd out, and may contain it's
    own boot sequence to setup things like interrupts and the stack space.

\end{enumerate}

It may be helpful to compare operation of the FSBL with the explanation
of the emulated UART and \SOCDIR{bin/upload-program.py} scripts
in \SECREF{flow:verilator:uart}.

