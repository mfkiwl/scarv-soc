
The board support package is a library which provides a software
abstraction to all of the hardware peripherals on the SoC.
It's source files are kept in \SOCDIR{src/bsp/}.
It is built as a static library which other programs can then link
against.

\begin{itemize}
\item The BSP is built using:

\begin{lstlisting}[language=bash,style=block]
> make bsp-lib
\end{lstlisting}

    This builds a static library, and places all build
    artifacts in \SOCDIR[\SOCWORK]{bsp/}.


\item The BSP is ``installed" to \SOCDIR[\SOCWORK]{bsp/} with the
    folder structure shown in Table \ref{tab:sw:bsp:organisation}.

\begin{table}[H]
\centering
\begin{tabular}{ll}
Directory & Description \\ \hline
\SOCWORK/bsp & Makefiles and linker scripts for building and running tests. \\
\hspace{1.0cm} lib/     & Contains the built static library. \\
\hspace{1.0cm} include/ & Header files for each library component. \\
\hspace{1.0cm} obj/     & Individual source object files. \\
\hspace{1.0cm} doc/     & Generated API documentation. \\
\end{tabular}
\caption{Organisation of the BSP install directory under \SOCWORK.}
\label{tab:sw:bsp:organisation}
\end{table}

\item To compile against the BSP, add
    \SOCDIR[\SOCWORK]{bsp/include}
    to the header search path using GCCs {\tt -I} flag, and add the
    \SOCDIR[\SOCWORK]{bsp/lib/libscarvsocbsp.a}
    to the list of input files when linking object files together.

\item The documentation for the BSP can be built using:

\begin{lstlisting}[language=bash,style=block]
> make bsp-docs
\end{lstlisting}

    This will place the generated HTML documentation in
    \SOCDIR[\SOCWORK]{bsp/doc}.
        
\item The BSP build can be cleaned up using

\begin{lstlisting}[language=bash,style=block]
> make bsp-clean
\end{lstlisting}

\end{itemize}

