
The inner subsystem contains two tightly coupled local memories:
A 1Kb ROM, which contains the First Stage Boot Loader,
A 64Kb RAM, which contains downloaded programs instructions and data.
All RTL code for the local memories is contained in \SOCDIR{rtl/mem}.
There are several different models of the local memories, each
used for a different purpose.

\begin{itemize}

\item \SOCDIR{rtl/mem/scarv\_soc\_bram\_dual\_sim.v} - 
    A simple simulation version, which is used with the Verilator
    flow. See \SECREF{flow:verilator} for more information.

\item \SOCDIR{rtl/mem/scarv\_soc\_bram\_dual\_synth\_xilinx\_xc7k.v} - 
    A Xilinx specific module, which uses the recommended language
    template to ensure BRAM inference during the synthesis flow.

\item \SOCDIR{rtl/mem/scarv\_soc\_bram\_dual\_synth\_yosys.v} - 
    BRAM model used in the Yosys synthesis flow.
    This is a placeholder for future work.

\item \SOCDIR{rtl/mem/scarv\_soc\_bram\_dual\_formal.v} - 
    A blackboxed BRAM module used by the formal verification flows.
    This module contains no {\em memory} in the traditional sense, and
    is constrained only to obey the BRAM signalling protocol.
    Any data which is read or written is {\em not} held consistently.
    Writes are effectively ignored.
    Reads can return {\em anything} which the solver deems nessesary
    in order to try and (dis-)prove an assertion or assumption.

\end{itemize}
