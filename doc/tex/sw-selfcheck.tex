
The selfchecking tests are used as simple smoke-tests to make sure
that large features are working in the general case.
They {\em do not} consitute a proper functional verification effort.

\subsubsection{Selfchecking Tests Bootloader}

The selfchecking tests use a dedicated bootloader.
In normal usage, programs are downloaded to the SoC via the
First Stage Bootloader (FSBL, \SECREF{sw:fsbl}).
For testing purposes though, this is slow and not useful.
Instead, the self-checking programs are directly loaded into
the RAM memories.

This means the normal FSBL is not needed, so it is replaced by the one in
\SOCDIR{verif/selfcheck/bootloader}.
All this bootloader does is zero the general purpose registers,
disable interrupts and jump to the main program RAM.
The compiled version of this bootloader forms the {\tt rom.hex}
file for each selfchecking test.

\subsubsection{Test listing}

The following is a list of the tests and a short description of
their purpose.

\begin{itemize}
\item {\tt 00\_example} - A skeleton test which does nothing but boot
    and pass. 
    Useful for developing new tests or understanding the make flow.

\item {\tt 01\_simple} -
    Computes Fibonacci numbers. Tests that general instruction
    code execution works and that the stack functions correctly.

\item {\tt 02\_ram\_rom\_data} -
    Checks that we can properly read and write to the RAM/ROM
    modules as appropriate.

\item {\tt 03\_data\_load\_error} -
    Checks that a data load to unmapped addresses correctly returns an
    error response.

\item {\tt 04\_data\_store\_error} -
    Checks that a data store to unmapped addresses correctly returns an
    error response.

\item {\tt 05\_instr\_bus\_error} - 
    Checks if trying to execute instructions from unmapped memory
    correctly returns an error response.

\item {\tt 06\_axi\_bridge} -
    Checks that the AXI bridge can be used to access the outer
    SoC layer.

\item {\tt 07\_axi\_uart} - 
    Tests the UART driver and emulated model.

\item {\tt 08\_rng}
    Checks that the random number generator instantiation has been
    correctly integrated with the SCARV-CPU.

\end{itemize}

\subsubsection{Building Tests}

The selfchecking tests are organised as shown in table
\ref{tab:sw:selfcheck:organisation}.
The following commands show how to build the tests.:

\begin{itemize}

\item The selfchecking test bootloader is build with:

\begin{lstlisting}[language=bash,style=block]
> make selfcheck-bootloader
\end{lstlisting}
    
    If needed, the bootloader is automatically (re-)built by any target
    which needs it.

\item An individual test can be built using:

\begin{lstlisting}[language=bash,style=block]
> make selfcheck-build-test_<X>
> make selfcheck-build-test_00_example
\end{lstlisting}

    where {\tt <x>} corresponds to one of the {\tt test\_X} folders
    in \SOCDIR{verif/selfcheck}.

\item Building any test places all build artifacts in
    \SOCDIR[\SOCWORK]{selfcheck/test\_*}.

    This includes the compiled elf file, disassembly, GTKWave
    annotations and simulation hex file images.
    It is the {\tt *.hex} files which are loaded by the simulations.

\end{itemize}

\begin{table}[H]
\centering
\begin{tabular}{ll}
Directory & Description \\ \hline
\SOCHOME/flow/selfcheck    & Makefiles and linker scripts for building and running tests. \\
\SOCHOME/verif/selfcheck   & Root directory for selfchecking test sources \\
\hspace{1.0cm} bootloader/ & Source code for the selfchecking test bootloader. \\
\hspace{1.0cm} share/      & Shared source code for commonly used functions. \\
\hspace{1.0cm} test\_XX\_X/& Contains all source code for test X. \\
\SOCWORK/selfcheck         & All generated build artifacts are put here. \\
\end{tabular}
\caption{Selfchecking tests directory organisation}
\label{tab:sw:selfcheck:organisation}
\end{table}


\subsubsection{Running Tests}

\begin{itemize}

\item Any test can be run using the make flow:

\begin{lstlisting}[language=bash,style=block]
> make selfcheck-run-test_<X>
> make selfcheck-run-test_00_example
\end{lstlisting}

    This will run the test, and display a pass or fail result.

\item By default, selfchecking tests do not dump waveform views.
    To turn this on, run with:

\begin{lstlisting}[language=bash,style=block]
> make SELFCHECK_WAVES=1 selfcheck-run-test_<X>
> make SELFCHECK_WAVES=1 selfcheck-run-test_00_example
\end{lstlisting}

    This will cause a {\tt test\_00\_example.vcd} file to be written into
    \SOCDIR[\SOCWORK]{/selfcheck/test\_00\_example}.

\item All of the selfchecking tests can be run using:

\begin{lstlisting}[language=bash,style=block]
> make selfcheck-run-all
\end{lstlisting}

    Again, the {\tt SELFCHECK\_WAVES=1} argument can be added to dump waves.

\begin{lstlisting}[language=bash,style=block]
> make SELFCHECK_WAVES=1 selfcheck-run-all
\end{lstlisting}

    It is recommended to run selfchecking tests each time an RTL
    change is made to the SoC as a sanity check, before running the
    formal verification flow.

\end{itemize}


\subsubsection{Adding New Tests}

\begin{itemize}
\item Create a new directory under \SOCDIR{verif/selfcheck/} named
    in the form {\tt test\_XX\_Y}, where {\tt XX} is the next available
    test number, and {\tt Y} is a short descriptive name.

\item All source files for the program should live in this directory.
    It is easiest to copy the example test program and extend it.

\item All C and assembly files in that folder are automatically included.
    The make flow performs whole program compilation. Essentially,
    every source file in the folder is fed to a single GCC invocation.

\item The folder path is automatically added to the header include
    search path.

\item Programs can also include the {\tt selfcheck.h} header, which
    grants access to all of the functions in\\
    \SOCDIR{verif/selfcheck/share}.

\item Pass or fail is determined by the return value of the C main function.
    Return $0$ for pass, or non-$0$ for failure.
\end{itemize}

